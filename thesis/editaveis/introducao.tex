\chapter[Introdução]{Introdução}
\label{sec:int}

Este captulo tem a finalidade de apresentar uma contextualização a cerca dos assunstos tratados neste trabalho, analisando os comportamentos de uma rede neural recursiva aplicada na predição do índice da bolsa de valores Brasileira. A partir da \nameref{sec:contex} estabelecemos o contexto da aplicação do projeto, levando a \nameref{sec:qp} do estudo, permitindo compreender a \nameref{sec:justificativa} e os principais Objetivos desse trabalho. Concluindo o capítulo apresentando a \nameref{sec:orgt}, com os demais capítulos que constitui essa monografia.

\section{Contextualização}
\label{sec:contex}


\section{Justificativa}
\label{sec:justificativa}

\section {Questão de Pesquisa}
\label{sec:qp}

Em consideração ao apresentado, esse trabalho de monografia visa responder a seguinte questão de pesquisa: \textit{É possível a utilização de redes neurais recorrentes para previsionar o indice da bolsa de valores brasileira?}

\section{Objetivos}
\label{sec:obj}


\subsection{Objetivo Geral}
\label{sec:objgeral}

Averiguar a viabilidade da utilização de uma rede neural recorrente para predizer o índice da bolsa de valores brasileira fundamentando uma análise de processamento digital de sinais de séries temporais.



\subsection{Objetivos Específicos}
\label{sec:objesp}

- Listar variaveis para trablhar

\section{Organização do Trabalho}
\label{sec:orgt}
